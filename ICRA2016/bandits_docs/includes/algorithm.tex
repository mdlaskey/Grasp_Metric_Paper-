\section{Correlated Multi-Armed Bandit Algorithm}
\seclabel{algorithm}
The Dex-Net 1.0 algorithm (see pseudocode below) optimizes $P_F$ over a set of candidate grasps on a new object $\mO$ using Multi-Armed Bandits (MABs) with correlated rewards~\cite{hoffman2013exploiting, pandey2007multi} and priors computed from Dex-Net 1.0.
We first generate a set of candidate grasps $\Gamma$ for object $\mO$ using the antipodal grasp sampling described in \secref{grasp-sampling} and predict a prior belief distribution for each grasp using the Dex-Net database $\mD$.
Next, we run MAB by selecting a grasp using Thompson sampling~\cite{laskey2015bandits, oberlin2015autonomously}, sampling from the uncertainty random variables, determining force closure for the grasp on the sampled variables as described in \secref{quality}, and updating a belief distribution on the $P_F$ for each grasp.
Finally, we rank the grasps in $\Gamma$ by the maximum lower confidence bound of the belief distribution, a conservative estimate of the $P_F$ of each grasp, and store the ranking in the database.
We use Thompson sampling to study the scaling effects for a fixed grasp selection method and plan to study other methods based on confidence bounds~\cite{kroemer2010combining, oberlin2015autonomously} or Gittins indices~\cite{laskey2015bandits} in future work.

\subsection{Belief Distribution Model}
\seclabel{belief}
Let $\mO$ denote the test object to label with the Dex-Net Algorithm, and let $\Gamma$ be the set of $N_g$ candidate grasps generated for $\mO$.
We define $F_{j} = F(\bg_{j}) \in \{0, 1\}$ as force closure on an evaluation of any grasp $\bg_{j} \in \Gamma$ from samples of object pose, gripper pose, and friction as described in \secref{quality}.
Under the model, $F_j$ is a Bernoulli random variable with probability of success $\theta_{j}= P_F(\bg_{j})$.
Since $\theta_j$ is unknown, the algorithm maintains a posterior Beta belief distribution on the Bernoulli parameter $\theta_{j}$ that is updated with every new observation of $F$, assigning increasingly high probability to the true $P_F$. 
The Beta distribution~\cite{hoffman2013exploiting, laskey2015bandits} is specified by shape parameters $\alpha > 0$ and $\beta > 0$:

\vspace{-4ex}
\begin{align*}
	\betadist(\alpha, \beta) = Z(\alpha, \beta) \theta_j^{\alpha-1} (1 - \theta_j)^{\beta-1}
\end{align*}
\noindent where $Z(\alpha, \beta)$ is a normalization constant.

\subsection{Predicting Grasp Quality Using Prior Data}
\seclabel{ccbps}
We use Continuous Correlated Beta Processes (CCBPs)~\cite{goetschalckx2011continuous, montesano2012active} to model correlations between the $P_F$ of grasps on different objects, which allows us to utilize prior grasp and object data from Dex-Net.
CCBPs model correlations between Bernoulli random variables in a Beta-Bernoulli process, which exist when the variables depend on common latent factors.
Two grasps on an object may have similar $P_F$ when they contact the object at similar locations, as evidenced by Lipschitz bounds on grasp wrench space metrics~\cite{pokorny2013c}.

A CCBP estimates the shape parameters for a grasp-object pair $\mY_{j} = (\bg_{j}, \mO_{i}) \in \mM$ using a normalized kernel function $k(\mY_p, \mY_q) : \mM \times \mM \rightarrow [0,1]$ that measures similarity between a pair of grasps and objects from the Grasp Moduli Space $\mM$.
The kernel approaches 1 as the arguments become increasingly similar and approaches 0 as the arguments become dissimilar.

We measure similarity using a set of feature maps $\phi_m: \mM \rightarrow \mathbb{R}^{d_m}$ for $m = 1, ..., 3$, where $d_m$ is the dimension of the feature space for each.
The first feature map $\phi_{1}(\mY) = (\bx, \bv, \| \rho_1 \|_2, \| \rho_2 \|_2)$ captures similiarity in the grasp parameters, where $\bx \in \bR^3$ is the grasp center, $\bv \in \bS^2$ is the grasp approach, and $\rho_i \in \bR^3$ is the $i$-th moment arm.
To capture local surface geometry, the second feature map $\phi_{2}(\mY) = \eta(\bg, \mO)$, where $\eta$ is the differential heightmap described in \secref{grasp-similarity}.
To capture global shape information, our third feature map $\phi_{3}(\mY) = \psi(\mO)$, where $\psi$ is our object similarity map described in \secref{object-similarity}.
Given the feature maps, we use the squared exponential kernel 
\begin{align*}
	k(\mY_p, \mY_q) &= \exp\left( - \frac{1}{2} \sum \limits_{m=1}^{3} \|\phi_m(\mY_p) - \phi_m(\mY_q)\|_{C_m}^2 \right).
\end{align*}
\noindent where $C_m \in \bR^{d_m \times d_m}$ is the inverse bandwidth for $\phi_m$ and $\| \by \|_{C_m} = \by^T C_m^T C_m \by$.
The bandwidths are set by maximizing the log-likelihood~\cite{goetschalckx2011continuous} of the true $P_F$ under the CCBP on a set of training data.

We form a prior belief distribution for each candidate grasp in $\Gamma$ based on its similarity to all grasps and objects from the Dex-Net 1.0 database $\mD$ as measured by the kernel~\cite{goetschalckx2011continuous}:
%\vspace{-2ex}
\begin{align}
	%p \left(\theta_{j} | \mD \right) &= \betadist\left( \alpha_{j,0}, \beta_{j,0} \right) \notag \\
	\alpha_{j,0} = \alpha_{0} & + \sum \limits_{i=1}^{N_o} \sum \limits_{k=1}^{N_g} k(\mY_{j}, \mY_{i, k}) S_{i,k} \label{eq:alpha-prior} \\
	\beta_{j,0} = \beta_{0} & + \sum \limits_{i=1}^{N_o} \sum \limits_{k=1}^{N_g}  k(\mY_{j}, \mY_{i,k}) (N_s - S_{i,k}) \label{eq:beta-prior}
\end{align}
\noindent where $\alpha_{0}$ and $\beta_{0}$ are prior parameters for the Beta distribution~\cite{laskey2015bandits} and $N_s$ is the number of times each grasp in $\mD$ was sampled to evaluate $P_F$.
In practice, we estimate the above sums using the $N_n$ nearest neighbors to $\mO$ in the object similarity KD-Tree described in \secref{object-similarity}.
Upon observing $F_{\ell}$ for grasp $\bg_{\ell}$ on iteration $t$, we update our belief for all other grasps on object $\mO$ by~\cite{goetschalckx2011continuous}:

\vspace{-2ex}
\begin{align}
	\alpha_{j,t} &= \alpha_{j,t-1} + k(\mY_{j}, \mY_{\ell}) F_{\ell} \label{eq:alpha} \\
	\beta_{j,t} &= \beta_{j,t-1} + k(\mY_{j}, \mY_{\ell}) (1 - F_{\ell})\label{eq:beta}.
\end{align}

\begin{algorithm}
{\small
    \SetAlgoLined
    {\bf Input:} Object $\mO$, Number of Candidate Grasps $N_g$, Number of Nearest Neighbors $N_n$, Dex-Net 1.0 Database $\mD$, Features maps $\psi$ and $\eta$,  Maximum Iterations $T$, Prior beta shape $\alpha_0$, $\beta_0$, Lower Bound Confidence $p$, Random Variables $\nu$, $\xi$, and $\gamma$ \\
    \KwResult{Estimate of the grasp with highest $P_F$, $\hat{\bg}^*$}
    
    \tcp{Generate candidate grasps and priors}
	$\Gamma$ = AntipodalGraspSample($\mO, N_g$) \;
	$\mA_0 = \varnothing, \mB_0 = \varnothing$\;
	\For{$\bg_k \in \Gamma$}{
		\tcp{Equations~\ref{eq:alpha-prior} and ~\ref{eq:beta-prior}}
		$\alpha_{k,0}, \beta_{k,0} =$ ComputePriors($\mO, \bg_k, \mD, N_n, \psi$)\; 
		$\mA_0 = \mA_0 \cup \{\alpha_{k,0}\}, \mB_0 = \mB_0 \cup \{\beta_{k,0}\}$\;
	}
	
	\tcp{Run MAB to Evaluate Grasps}
	\For{$t = 1, .., T$}{
		$j = $ ThompsonSample($\mA_{t-1}, \mB_{t-1}$)\;
		$\hat{\nu}, \hat{\xi}, \hat{\gamma} =$ SampleRandomVariables($\nu, \xi, \gamma)$\;
		$F_{j} = $ EvaluateForceClosure($\bg_j, \mO, \hat{\nu}, \hat{\xi}, \hat{\gamma}$)\;
		\tcp{Equations~\ref{eq:alpha} and ~\ref{eq:beta}}
		$\mA_{t}, \mB_{t} = $ UpdateBeta($j, F_j, \Gamma$)\; 
		$\bg_{t}^* = $MaxLowerConfidence($\mA_{t}, \mB_{t}, p$)\;
	}
	return $\bg_T^*$\;
    \nonl {\bf Dex-Net 1.0 Algorithm}: Robust Grasp Planning Using Multi-Armed Bandits with Correlated Rewards
}
\end{algorithm}

