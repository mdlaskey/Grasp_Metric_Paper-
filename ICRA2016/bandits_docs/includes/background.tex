\section{Continuous Correlated Beta Processes}
\seclabel{background}

Finding an approximate solution of Equation~\ref{eq:objective} using Multi-Armed Bandits requires a model of the reward distribution of each arm.
We use Continuous Correlated Beta Processes (CCBPs) to model the probability of force closure of grasps as a Bernoulli random variable with correlations between grasps.
We refer the reader to Goetschalckx et al.~\cite{} for a detaled description of CCBPs.
In practice, correlations may occur between grasps that contact the shape in similar locations may have similar probabilities of force closure.
Correlations may also occur between grasps on different shapes, such as grasping the handle of a mug and the handle of a teapot.

\subsection{Beta-Bernoulli Processes}

For clarity we first review Beta-Bernoulli Processes, the uncorrelated version of CCBPs. 
Let $\Gamma = \{g_k\}_{k=1}^K$ denote our discrete set of $K$ candidate grasps.
Let $F(\bg_k) \in \{0, 1\}$ denote the occurence of force closure for a grasp $\bg_k$.
We model $F(\bg_k)$ as a Bernoulli random variable with probability of success $\theta_k = P_F(\bg_k)$.
Since we do not know the value of $\theta_k$ for each grasp, we can specify a distribution of $\theta_k$ based on our prior belief about the likelihood of force closure.
A common choice for a prior on the Bernoulli parameter $\theta_k$ is the Beta distribution, which is specified by shape parameters $\alpha > 0$ and $\beta > 0$:

\vspace{-2ex}
\begin{align*}
	\betadist(\alpha, \beta) = \frac{1}{B(\alpha, \beta)} \theta_k^{\alpha-1} (1 - \theta_k)^{\beta-1}
\end{align*}

\noindent This combination of indepdent Bernoulli random variables with Beta priors is known as a {\it Beta-Bernoulli process}.

Let our prior distribution on $\theta_k$ be a beta distribution with parameters $\alpha_{k,0}$, $\beta_{k,0}$.
Then after observing force closure for $t$ samples $F_{1}(g_k), ..., F_{t}(g_k)$ from our uncertainty model, our posterior distribution on the Bernoulli success probability $\theta_k$ is obtained using Bayes rule.
This yields~\cite{laskey2015bandits}

\vspace{-2ex}
\begin{align*}
	p \left(\theta_k | F_{1}(g_k), ..., F_{t}(g_k) \right) &= \betadist\left( \alpha_{k,t}, \beta_{k,t} \right) \\
	\alpha_{k,t} &= \alpha_{k,0} + \sum \limits_{i=1}^t F_{i}(g_k) \\
	\beta_{k,t} &= \beta_{k,0} + \sum \limits_{i=1}^t (1 - F_{i}(g_k))
\end{align*}

\noindent Therefore the belief distribution on the probability of force closure $\theta_k$ can be updated by incrementing the value of $\alpha$ and $\beta$ for each force closure observed.
The expected value of $\theta_k$ after the observations is

\begin{align*}
	\mathbb{E} \left[ \theta_k | F_{1}(g_k), ..., F_{t}(g_k) \right] &= \frac{\alpha_{k,t}}{\alpha_{k,t} + \beta_{k,t}} \\
	&= \frac{\mbox{\#Successes} +	
	\alpha_{k,0}}{\mbox{\#Trials}+\alpha_{k,0}+\beta_{k,0}}.
\end{align*}

\subsection{Continuous Correlated Beta Processes}

Continuous Correlated Beta Processes(CCBPs) were independently developed by Goetschalckx et al.~\cite{goetschalckx2011continuous} and Montesano and Lopes~\cite{montesano2012active} to model correlations between the Bernoulli random variables in a Beta-Bernoulli process, which may lead to faster convergence in Multi-Armed Bandit problems~\cite{chu2011contextual}.
Such correlations may exist when the Bernoulli random variables depend on common latent factors.
For example, two probability of force closure variables $\theta_k$ may be correlated when two grasps have similar poses or when the surface geometry near the points at which they contact an object are similar.

A CCBP is a Beta-Bernoulli process with an additional parameter $k(\bg_i, \bg_j) : \mathcal{G} \times \mathcal{G} \rightarrow [0,1]$, called the kernel, that measures similarity between two grasps.
The kernel approaches 1 as the arguments become increasingly similar and approaches 0 as the arguments become dissimilar.
One common choice of kernel is the squared exponential kernel 
\begin{align*}
	k(\bg_i, \bg_j) &= \exp\left( -d(\bg_i, \bg_j)^2 \right)
\end{align*}
\noindent where $d(\cdot): \mathcal{G} \times \mathcal{G} \rightarrow \mathbb{R}_{+}$ is a distance metric between grasps.
A common choice of distance metric is
\begin{align*}
	d(\bg_i, \bg_j) &= \left( \phi(\bg_i) - \phi(\bg_j) \right)^T C^{-1} \left( \phi(\bg_i) - \phi(\bg_j)\right)
\end{align*}
\noindent where $\phi(\cdot) : \mG \rightarrow \mathbb{R}^m$ is an $m$-dimensional feature representation of grasps and $C \in \mathbb{R}^{m \times m}$ is the {\it kernel bandwidth}.
The bandwidth $C$ scales dimensions of the feature space to mark some dimensions as being more or less relevant for measuring similarity between grasps.

Given a kernel, on each observation we update the belief of each grasp proportional to how similar it is to the observed grasp as measured by the kernel. 
Let $F_{1}(\bg_{I(1)}), ..., F_{t}(\bg_{I(t)})$ be $t$ observations of force closure from samples of our uncertainty models for grasps $g_{I(1)}, ..., g_{I(t)}$, where $I(j)$ is the index of the grasp sampled at time $j$.
Then the CCBP posterior update for $\theta_k$ is~\cite{}:

\vspace{-2ex}
\begin{align*}
	p \left(\theta_k | F_{1}(g_{I(1)}), ..., F_{t}(g_{I(t)}) \right) &= \betadist\left( \alpha_{k,t}, \beta_{k,t} \right) \\
	\alpha_{k,t} = \alpha_{k,0} + \sum \limits_{j=1}^t k(g_k, g_{I(j)}) F_{i}(g_{I(j)}) \\
	\beta_{k,t} = \beta_{k,0} + \sum \limits_{=1}^t k(g_k, g_{I(j)}) (1 - F_{i}(g_{I(j)}))
\end{align*}

\noindent Intuitively, this allows observations of a grasp to constitute "effective" observations of other grasps.
