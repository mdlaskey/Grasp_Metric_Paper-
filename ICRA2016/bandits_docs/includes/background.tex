\section{Continuous Correlated Beta Processes}
\seclabel{background}

We use Continuous Correlated Beta Processes (CCBPs) to model correlations in the probability of force closure between two grasps.
For example, two grasps that contact the shape in similar locations may have similar probabilities of force closure.
We refer the reader to Goetschalckx et al.~\cite{} for a detaled description of CCBPs.

\subsection{Beta-Bernoulli Processes}

Consider a set of $K$ grasps $\Gamma = \{g_k\}_{k=1}^K$.
Then force closure for each grasp $F(g_k) \in \{0, 1\}$ is a Bernoulli random variable with probability of success $\theta_k = P_F(g_k)$.
Since we do not know the value of $\theta_k$ for each grasp, we can specify a distribution of $\theta_k$ based on our prior belief about the likelihood of force closure.
A common choice for a prior on the Bernoulli parameter $\theta_k$ is the Beta distribution, which is specified by shape parameters $\alpha > 0$ and $\beta > 0$:

\vspace{-2ex}
\begin{align*}
	\betadist(\alpha, \beta) = \frac{1}{B(\alpha, \beta)} \theta_k^{\alpha-1} (1 - \theta_k)^{\beta-1}
\end{align*}

\noindent This combination of indepdent Bernoulli random variables with Beta priors is known as a {\it Beta-Bernoulli process}.

Let our prior distribution on $\theta_k$ be a beta distribution with parameters $\alpha_{k,0}$, $\beta_{k,0}$.
Then after observing force closure for $t$ samples $F_{1}(g_k), ..., F_{t}(g_k)$ from our uncertainty model, our posterior distribution on the Bernoulli success probability $\theta_k$ is obtained using Bayes rule.
This yields~\cite{laskey2015bandits}

\vspace{-2ex}
\begin{align*}
	p \left(\theta_k | F_{1}(g_k), ..., F_{t}(g_k) \right) &= \betadist\left( \alpha_{k,t}, \beta_{k,t} \right) \\
	\alpha_{k,t} &= \alpha_{k,0} + \sum \limits_{i=1}^t F_{i}(g_k) \\
	\beta_{k,t} &= \beta_{k,0} + \sum \limits_{i=1}^t (1 - F_{i}(g_k))
\end{align*}

\noindent Therefore the belief distribution on the probability of force closure $\theta_k$ can be updated by incrementing the value of $\alpha$ and $\beta$ for each force closure observed.
The expected value of $\theta_k$ after the observations is

\begin{align*}
	\mathbb{E} \left[ \theta_k | F_{1}(g_k), ..., F_{t}(g_k) \right] = \frac{\alpha_{k,t}}{\alpha_{k,t} + \beta_{k,t}} = \frac{\mbox{\#Successes} +	
	\alpha_{k,0}}{\mbox{\#Trials}+\alpha_{k,0}+\beta_{k,0}}.
\end{align*}

\subsection{Continuous Correlated Beta Processes}

Continuous Correlated Beta Processes(CCBPs) were introduced by Goetschalckx et al.~\cite{} to model correlations between the Bernoulli random variables in a Beta-Bernoulli process, which may lead to faster convergence in Multi-Armed Bandit problems~\cite{}.
Such correlations may exist when the Bernoulli random variables depend on common latent factors.
For example, two probability of force closure variables $\theta_k$ may be correlated when two grasps have similar poses or when the surface geometry near the points at which they contact an object are similar.

A CCBP is a Beta-Bernoulli process with an additional parameter $k(g_i, g_j) : \mathcal{G} \times \mathcal{G} \rightarrow [0,1]$, called the kernel, that measures similarity between two grasps.
The kernel should approach 1 as the arguments become increasingly similar and approach 0 as the arguments become dissimilar.
Let $F_{1}(g_{I(1)}), ..., F_{t}(g_{I(t)})$ be $t$ observations of force closure from samples of our uncertainty models for grasps $g_{I(1)}, ..., g_{I(t)}$, where $I(j)$ is the index of the grasp sampled at time $j$.
Then the CCBP posterior update for $\theta_k$ is~\cite{}:

\vspace{-2ex}
\begin{align*}
	p \left(\theta_k | F_{1}(g_{I(1)}), ..., F_{t}(g_{I(t)}) \right) &= \betadist\left( \alpha_{k,t}, \beta_{k,t} \right) \\
	\alpha_{k,t} &= \alpha_{k,0} + \sum \limits_{j=1}^t k(g_k, g_{I(j)}) F_{i}(g_{I(j)}) \\
	\beta_{k,t} &= \beta_{k,0} + \sum \limits_{=1}^t k(g_k, g_{I(j)}) (1 - F_{i}(g_{I(j)}))
\end{align*}

\noindent Therefore grasps can constitute "effective" observations of other grasps if the two grasps are similar according to the kernel function.
