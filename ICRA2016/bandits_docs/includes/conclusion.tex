\section{Discussion and Future Work}
\seclabel{conclusion}
We presented Dexterity Network 1.0 (Dex-Net), a new dataset and associated algorithm to study the scaling effects of Big Data and Cloud Computation on robust grasp planning with binary grasp quality metrics.
The algorithm uses a Multi-Armed Bandit model with correlated rewards to leverage prior grasps and 3D object models and Multi-View Convolutional Neural Networks (MV-CNNs), a new deep learning method for 3D object classification, as a similarity metric between objects. 
In experiments, the Google Cloud Platform allowed Dex-Net 1.0 to simultaneously run up to 1,500 virtual machines, reducing experiment runtime by approximately three orders of magnitude.
Experiments suggest that prior data can speed robust grasp planning by a factor of 2 and that average grasp quality increases with the number of similar objects in the dataset.

In future work, we will develop metrics and algorithms for pre-computing a set of grasps that adequately ``cover" each object from a variety of accessibility conditions (depending on pose and occlusions).
We also plan to evaluate the Dex-Net 1.0 algorithm in physical experiments and study convergence rates when using various analytic (e.g. force closure) and simulation-based grasp quality labels from Dex-Net 1.0 to pre-predict physical grasp successes.
We will also explore Deep Learning~\cite{krizhevsky2012imagenet} to estimate grasp quality from surface depthmaps and rendered object images for further accelerating convergence.
We also hope to release subsets of Dex-Net 1.0 with an open-source API to explore robust grasping as a service (RGaaS).

