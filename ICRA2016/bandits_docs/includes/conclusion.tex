\section{Discussion and Future Work}
\seclabel{conclusion}
We developed the Dex-Net 1.0 algorithm using a Multi-Armed Bandit model with correlated rewards to reduce the number of samples required to plan robust parallel-jaw grasps by leveraging prior grasps and 3D object models.
To study scaling effects, we introduced the Dexterity Network (Dex-Net) 1.0, a dataset of approximately 10,000 3D object models and 2.5 million parallel-jaw grasps, and MV-CNNs~\cite{aubry2015understanding, su2015multi} to efficiently index similar objects.
%For efficient indexing of similar objects in Dex-Net, we used multi-view Convolutional Neural Networks (MV-CNNs)~\cite{su2015multi}, a state-of-the-art deep learning method of object classification.
Our experiments suggest that increasing the amount of prior knowledge of objects and the quality of grasps on each object enables up to a $10\times$ speedup in the time to compute a grasp with high probability of force closure for a new object, and also suggest that convergence is fastest when the object has a set of geometrically similar nearest neighbors in the database.

One current shortcoming of our method is that it relies on a set of features manually designed to predict grasp quality for a grasp and object, such as heightmaps and CNNs that predict object category.
Future work will leverage recent developments in feature learning with Deep Neural Networks~\cite{krizhevsky2012imagenet} to optimize grasp and object representations directly based on the outcome of grasp evaluations from bandit algorithms.
We are also interested in an end-to-end approach that integrates research on deep learning to map directly from images to motor torques that control a grasp.
To do so, we will use supervision from the data in Dex-Net to study prediction of object identity and pose directly from depth images~\cite{aubry2015understanding} and deep learning for control policies to map planned grasps on objects to motor torques~\cite{levine2015end}. 
%We will use supervision from the data in Dex-Net to study extensions of our MV-CNNs to predict 3D object model and pose from simulated depth images of objects on a planar work surface and to research deep learning approaches that map from 3D objects directly to grasp or manipulation policies, with the goal of a future system that maps directly from images to motor torques.

In future work, we will also add grasp and object data to Dex-Net, such as more 3D models, task constraints, the outcomes of physical trials, or human annotations.
%This presents a challenge because the number of labels per object and grasp may be smaller that with probability of force closure due to the time cost involved.
%Thus we will study using functional map networks to establish consistent maps between objects to share sparse labels throughout Dex-Net~\cite{huang2013fine}.
We will also explore additional analytic grasp metrics to Dex-Net such as cages, different grasp selection policies such as Gittins indices, and extensions to plan a set of grasps that ``cover" an object surface to increase the chances that a grasp is reachable in cluttered environments.
We will also include uncertainty in object shape resulting from noisy point clouds~\cite{mahler2015gp} to future models.
Our goal is to release Dex-Net in the future as an open-access project with an open-source API for labelling objects in the Cloud with analytic quality metrics or simulation outcomes and integrating Dex-Net with physical robots.

