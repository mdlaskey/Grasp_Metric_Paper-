\section{Discussion and Future Work}
\seclabel{conclusion}
We presented Dexterity Network 1.0 (Dex-Net), a new dataset and associated algorithm to study the scaling effects of Big Data and Cloud Computation on robust grasp planning.
The algorithm uses a Multi-Armed Bandit model with correlated rewards to leverage prior grasps and 3D object models and Multi-View Convolutional Neural Networks (MV-CNNs), a new deep learning method for 3D object classification, as a similarity metric between objects.
In experiments, the Google Cloud Platform allowed Dex-Net 1.0 to simultaneously run up to 1,500 virtual machines, reducing experiment runtime by three orders of magnitude.
Experiments suggest that prior data can speed robust grasp planning by a factor of 2 and that average grasp quality increases with the number of similar objects in the dataset.
We reported on sensitivity to varying similarity metrics and pose and friction uncertainty levels.

In future work, we will develop metrics to pre-compute grasps that adequately ``cover" each object from a variety of accessibility conditions (depending on pose and occlusions).
We will also explore how Deep Learning~\cite{krizhevsky2012imagenet} can be used in other parts of a grasp planning pipeline, for example to recognize object pose and shape from images~\cite{aubry2015understanding}, to learn grasp and object features robust to shape variation using prior evaluations from bandit algorithms, and perhaps even to determine motor torques based on images and precomputed grasps~\cite{levine2015end}.
We also hope to release subsets of Dex-Net 1.0 with an open-source API to explore robust grasping as a service (RGaaS).

