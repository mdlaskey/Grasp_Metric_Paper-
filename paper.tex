%%%%%%%%%%%%%%%%%%%%%%%%%%%%%%%%%%%%%%%%%%%%%%%%%%%%%%%%%%%%%%%%%%%%%%%%%%%%%%%%
%2345678901234567890123456789012345678901234567890123456789012345678901234567890
%        1         2         3         4         5         6         7         8

\documentclass[letterpaper, 10 pt, conference]{ieeeconf}  % Comment this line out if you need a4paper

%\documentclass[a4paper, 10pt, conference]{ieeeconf}      % Use this line for a4 paper

\IEEEoverridecommandlockouts                              % This command is only needed if 
                                                          % you want to use the \thanks command

\overrideIEEEmargins                                      % Needed to meet printer requirements.

% See the \addtolength command later in the file to balance the column lengths
% on the last page of the document

% The following packages can be found on http:\\www.ctan.org
%\usepackage{graphics} % for pdf, bitmapped graphics files
%\usepackage{epsfig} % for postscript graphics files
%\usepackage{mathptmx} % assumes new font selection scheme installed
%\usepackage{times} % assumes new font selection scheme installed
%\usepackage{amsmath} % assumes amsmath package installed
%\usepackage{amssymb}  % assumes amsmath package installed



\usepackage{amsmath,amssymb}
\usepackage{tikz,hyperref,graphicx,units,subfig}
\usepackage{sidecap,wrapfig}
\usepackage[ruled,vlined]{algorithm2e}
\DeclareMathOperator*{\argmin}{arg\,min}
\DeclareMathOperator*{\argmax}{arg\,max}
\newcommand{\abs}[1]{\lvert#1\rvert} 
\newcommand{\norm}[1]{\lVert#1\rVert}
%\newcommand{\suchthat}{\mid}
\newcommand{\suchthat}{\ \big|\ }
\newcommand{\bd}{\mathbf{d}}
\newcommand{\bn}{\mathbf{n}}
\newcommand{\bp}{\mathbf{p}}
\newcommand{\bw}{\mathbf{w}}
\newcommand{\by}{\mathbf{y}}
\newcommand{\bx}{\mathbf{x}}
\newcommand{\bz}{\mathbf{z}}
\newcommand{\bbf}{\mathbf{f}}
\newcommand{\bzero}{\mathbf{0}}
\newcommand{\bG}{\mathbf{G}}
\newcommand{\bA}{\mathbf{A}}
\newcommand{\bW}{\mathbf{W}}
\newcommand{\bX}{\mathbf{X}}
\newcommand{\mX}{\mathcal{X}}
\newcommand{\mD}{\mathcal{D}}
\newcommand{\mN}{\mathcal{N}}
\newcommand{\mW}{\mathcal{W}}
\newcommand{\mF}{\mathcal{F}}
\newcommand{\bZ}{\mathbf{Z}}

\newcommand{\bfc}{W}
\newcommand{\Qinf}{Q_{\infty}}
\newcommand{\st}[1]{_\text{#1}}
\newcommand{\rres}{r\st{res}}
\newcommand{\pos}[1]{(#1)^+}
\newcommand{\depth}{\operatorname{depth}}
\newcommand{\dist}{\operatorname{dist}}
\newcommand{\convhull}{\operatorname{ConvexHull}}
\newcommand{\minksum}{\operatorname{MinkowskiSum}}



\title{\LARGE \bf
Grasp Metric for Shape Uncertainity with Gaussian Proccess Implicit Surface Representation (Not Finished Work) }


\author{Michael Laskey*, Zoe McCarthy*, Sachin Patil, Pieter Abbeel, and Ken Goldberg}% <-this % stops a space

\newtheorem{theorem}{Theorem}

\begin{document}



\maketitle
\thispagestyle{empty}
\pagestyle{empty}


%%%%%%%%%%%%%%%%%%%%%%%%%%%%%%%%%%%%%%%%%%%%%%%%%%%%%%%%%%%%%%%%%%%%%%%%%%%%%%%%



%%%%%%%%%%%%%%%%%%%%%%%%%%%%%%%%%%%%%%%%%%%%%%%%%%%%%%%%%%%%%%%%%%%%%%%%%%%%%%%%
\section{Introduction}

\vspace{10pt}
 A number of metrics have been proposed to evaluate form and force closure with scalar quality measures for grasping \cite{bicchi2000}. However, only recently have people started looking into a metric's robustness to uncertainty. Prior work by Zheng et al \cite{zheng2005}, looked at how to efficiently include uncertainty in friction coefficient and movement of gripper arm. However, they assumed a known surface of the object. In robotics today, where our objects are represented as noisy point clouds, shape uncertainty is a very common problem \cite{singhbigbird}. \\
 
 We use Gaussian Processes \cite{rasmussen2006} to convert the discrete mesh into an implicit surface with a measurement of uncertainty about the shape. Our contribution extends wrench-space Ferrari-Canny force closure quality measures \cite{ferrari1992}, which aims to maximize the disturbance that can be resisted given bounds on the contact forces, to incorporate shape uncertainty. We use recent results on a Lipschitz  constant for the Ferrari-Canny metric \cite{pokorny2013classical}, and prove a tight probabilistic bound on the change in grasp quality under shape uncertainty. We also provide an efficient way to calculate the distributions on the grasp parameters and include a way to update them quickly with new observations. We work in the general case where friction is at the contact points. 


\section{Gaussian Process Primer}
 Gaussian processes (GPs) are widely used in machine learning as a nonparametric regression method for estimating continuous functions from sparse and noisy data \cite{rasmussen2006}.
In a GP, a training set consists of input vectors $\mX = \{\bx_1, \ldots, \bx_n\}, ~\bx_i \in \mathbb{R}^d$, and corresponding observations $\by = \{y_1, \ldots, y_n\}$.
The observations are assumed to be noisy measurements from the unknown target function $f$:
\begin{equation}
y_i = f(\bx_i) + \epsilon,
\end{equation}
where $\epsilon \sim \mN(0,\sigma^2)$ is Gaussian noise in the observations.
A zero-mean Gaussian process is completely specified by a covariance function $k(\cdot,\cdot)$, also referred to as a kernel.
Given the training data $\mD = \{\mX, \by\}$ and covariance function $k(\cdot,\cdot)$, the posterior density $p(f_*|\bx_*,\mD)$ at a test point $\bx_{*}$ is shown to be \cite{rasmussen2006}:
\begin{equation}
  p(f_*|\bx_*,\mD) 
  \sim 
  \mN\big(k(\mX,\bx_*)^{\intercal}(K + \sigma_{\epsilon}^2I)^{-1}\by,
\end{equation}
\[
  k(\bx_*,\bx_*)-k(\mX,\bx_*)^{\intercal}(K+\sigma^2I)^{-1}k(\mX,\bx_*)\big), \label{eq:GPposterior}
\]
where $K \in \mathbb{R}^{n \times n}$ is a matrix with entries $K_{ij} = k(\bx_i,\bx_j)$ and $k(\mX,\bx_*) = [k(\bx_1,\bx_*),\ldots,k(\bx_n,\bx_*)]^{\intercal}$. 

The choice of kernel is application-specific, since the function $k(\bx_i,\bx_j)$ is used as a measure of correlation between states $\bx_i$ and $\bx_j$.
A common choice is the squared exponential kernel:
\begin{equation}
k(\bx_i,\bx_j) 
=
\nu^2\exp(-\frac{1}{2}(\bx_i - \bx_j)^{\intercal}\Lambda^{-1}(\bx_i - \bx_j))
\end{equation}
where $\Lambda= \text{diag}(\lambda_1^2,\ldots,\lambda_d^2)$ are the characteristic length scales of each dimension of $\bx$ and $\nu^2$ describes the variability of $f$.
The vector of hyper-parameters $\boldsymbol{\theta} = \{\sigma,\nu,\lambda_1,\ldots,\lambda_d\}$ is chosen or optimized during the training process by minimizing the log likelihood $p(\by|\mX,\boldsymbol{\theta})$ \cite{rasmussen2006}.

%\section{Related Work}

\section{Problem Definition}
Given a grasp $G$ on an object, we can define it by the following tuple $G = \lbrace c_1,...,c_m,n_1,...,n_m,z,\tau\rbrace$. We have a set $I$ of $m$ contacts on the object where $i \in I$ contact is located at $c_i$ with surface normal $n_i$.
The object has a center of mass $z$ and friction coefficient $\mu$.
We demonstrate that one can efficiently compute a closed form distribution for $c_i$,$n_i$ and $z$.
We note though that our metric assumes a known $\mu$ or friction coefficient. \\


 For the following derivations we use the following notation.
 $\theta(x) = \lbrace \mu(x),\Sigma(x) \rbrace$, hence $\theta(x)$ is a tuple consisting of the mean and covariance functions given by the trained GPIS.
We further assume that the contacts on the gripper approach along a line of action: $x = \gamma(t): R \rightarrow R^d$. The set of all elements in the range $[a,b]$ will be denoted by T. 
% TODO: Michael, I changed this from voxel grid because we're modeling a continuous space not a voxel grid, the voxel grid willbe an artifact of our integration/sampling strategies.
We assume a bounded rectangular workspace $\mathcal{R}$.
The line segment has endpoints $a,b$ that are defined as the start of the gripper and the intersection of the line with the end of the workspace respectively, as shown in Fig.
 \ref{fig:line_of_action}.
 % TODO: Draw a box around the workspace to show that b intersects the end of the workspace and draw a gripper approaching from a.  Also make A and B lowercase in the figure.
 We also define the function $f(x): \mathbb{R}^d \rightarrow \mathbb{R}$ as $f(x) = \mu(x)$ and an implicit surface $\mathcal{S} = \{ x \ | \ f(x) = 0 \}$.
% TODO: Michael, did I fill this in correctly? It was left blank initially.

\begin{figure}[ht!]
\centering
\includegraphics[scale = 0.3]{figures/Slide1.jpg}
\caption{Parameterized Line of Action along a Object}
\vspace*{-10pt}
\label{fig:line_of_action}
\end{figure}


\section{Distribution of Grasp Parameters}


\subsection{Distribution on Contact Points} The probability along the line $\gamma(t)$ is given by the following:

\begin{equation}
P(f(\gamma(a:b))|\theta(\gamma(a:b))) 
=
\mN(\mu_T,\Sigma_T)
\end{equation}



This gives the distributions along the entire line of action, however we want to compute the joint probability   $p(c_i=k) = p(f(\gamma(t))=0, f(\gamma(t-1:0))\neq 0)$. Hence we want to know at a point what the distribution  is that it is at the surface and no points before it on line are on the surface as well, this avoids the problem of the line of action intersecting multiple points on the surface. We now derive a way to compute an upper-bound on this distribution 

\begin{align*}
p(c_i = k) &= p(f(\gamma(t)) = 0)\\
               &*p(f(\gamma(t-1:0)) \neq 0 | f(\gamma(t)) = 0)
\end{align*}

Using the first product in the equation can be computed easily using the marginalization of a multivariate Gaussian distribution and the second one can be rewritten by conditioning the distribution \cite{petersen2008matrix}. 

\begin{align*}
p(f(\gamma(t-1:0)) \neq 0 | f(\gamma(t)) = 0) = p_c(f(\gamma(t-1:0)))
\end{align*}

\begin{align*}
p_c(f(\gamma(t-1:0))) &= p(f(\gamma(t-1)) \neq 0) \\
                                  &*p(f(\gamma(t-1:0)) \neq 0 | f(\gamma(t-1) \neq 0))
\end{align*}

It is now clear that the following can be said

\begin{align*}
p_c(f(\gamma(t-1:0))) &\leq p(f(\gamma(t-1)) \neq 0)\\
                                  &* p(f(\gamma(t-1:0)) \neq 0))
\end{align*}

We can furthermore conclude 

\begin{align*}
p_c(f(\gamma(t-1:0))) &\leq p(f(\gamma(t-1)) \neq 0) \\
				   &*..*p(f(\gamma(0)) \neq 0))
\end{align*}

Or in other words the independence assumption provides an upper bound on the distribution, which would otherwise be intractable to calculate. We can know write the conservative approximation to the distribution as follows: 

\begin{align*}
p(c_i = \gamma(t)) &\leq p(f(\gamma(t-1)) \neq 0) \\
				   &*..*p(f(\gamma(0)) \neq 0))
				   &*p(f\gamma(t) = 0)
\end{align*} 

\subsection{Distribution on Surface Normals} 
The distribution of surface normals $P(n_i = k)$ can be calculate as follows.
First we assume that some function exists $h(x) = \lbrace \mu_{\nabla}(x), \Sigma_{\nabla}(x) \rbrace$, hence given a point $x$ it returns the parameters for a Gaussian distribution around the gradient.
this function can be computed via learning the gradient \cite{gradient} or analytical differentiation of $f(x)$.
We note that both methods yield a Gaussian distribution.
We now demonstrate how to marginalize out the contact distribution and compute $P(n_i = k)$\\

From our distribution on contact points and Bayes rule we can compute the following: 

\begin{equation}
p(c_i = \gamma(g), n_i = k) = p(n_i = k | h(\gamma(t)))*p(c_i = \gamma(t))
\end{equation}

Now we can marginalize out the distribution on contacts:

\begin{equation}
P(n_i = k) = \int_T  p(n_i = k | h(\gamma(t)))*p(c_i = \gamma(t)) dg
\end{equation}

We then discretize the curve and approximate the integral by a sum: 

\begin{equation}
P(n_i = k) = \sum_T  p(n_i = k | h(\gamma(t)))*p(c_i = \gamma(t)) dg
\end{equation}


Thus, $P(n_i = k)$ is a multi-modal distributions composed of Gaussians sum together. 

\subsection{Distribution on Center of Mass} 

We define the quantity $\mathcal{D}(x) = \int_{-\infty}^{0} p(f(x) =  s \ | \ \theta(x)) ds$ and note that it is equal to the probability that x is interior to the surface under the current observations.
We assume that the object has uniform mass density and then $\mathcal{D}(x)$ is the expected mass density at x.
Then we can find the expected center of mass as:

\begin{equation}
  \bar{z} 
  =
  \frac
    {\int_{\mathcal{R}}x \mathcal{D}(x) dx}
    {\int_{\mathcal{R}}  \mathcal{D}(x) dx}
\end{equation}

which can be approximated by sampling $\mathcal{R}$ uniformly in a voxel grid and approximating the spatial integral by a sum.


\section{Probabilistic Bound on Grasp Metric}
Following recent work on proving a Lipschitz bound on the Ferrari-Canny Metric \cite{pokorny2013classical}, we prove that an extension is possible to give a probabilistic bound on the change in grasp quality.
We rewrite the results here:

\subsection{Prior Work}

We follow the notation of \cite{pokorny2013classical} except that their $\mu$ is our $\tau$ since we use $\mu$ for means.  
We re-state some of their central theorems for our use here.
For more detail, refer there.
$Q(g)$ is the exact $L^1$ grasp quality.
It is denoted by the following 

\begin{align}
  Q(g) &= \mbox{max}(0,q(g)) = -d(0,\mbox{Conv}(\{0\} \cup S(g))\\
-d(0,S) &= min_{||z|| = 1} h_{S(z)}\\
h_{S(z)} &= \mbox{sup}_{s\in S}\langle s,z\rangle\\
\end{align}

We denote the Ferrari-Canny version, which approximates the friction cone by a linearized set of wrenches\cite{ferrari1992}, as $Q^-_l(g)$.
The next theorem shows that the linearized wrench set used in Ferrari-Canny calculates a lower bound on $Q(g)$.\\

\begin{theorem}
  \cite{pokorny2013classical}
For any grasp g, we have $0 \leq Q_l^-(g) \leq Q(g)$.
Furthermore, $||Q(g) - Q^-_l(g)|| \rightarrow 0$ as $l \rightarrow \infty$ when $Q_l^-(g)$ is computed using a uniform approximation of the friction cones with $l$ edges. \\
\end{theorem}

The next theorems are used to show that $Q$ is Lipschitz continuous.

\begin{theorem}
\label{lemma35}
  \cite{pokorny2013classical}
For $w \in \mathbb{R}^3$, we have, for $n \in \mathbb{S}^2$ and for friction coefficient $\mu > 0$, 

\begin{align}
\mbox{sup}_{x \in C(n)} \langle x,w \rangle = \langle n,w \rangle + \tau||n \times w||
\end{align}

Hence, for $u = (a,b) \in \mathbb{R}^3 \times \mathbb{R}^3 = \mathbb{R}^6$, we have 

\[
h_{W_i(g)}(a,b) =
 \langle n_i,a+b\times(c_i-z)\rangle +
\]
\begin{align}
 +\tau ||n_i \times (a+b\times(c_i-z))||
\end{align}

\end{theorem}

\begin{theorem}
  \cite{pokorny2013classical}
We have \\

\begin{align}
q(g) 
=
\min_{u\in \mathbb{R}^6, ||u|| =1} h_{S(g)}(u) 
=
\end{align}
\[
\min_{u\in \mathbb{R}^6, ||u|| =1} \max_{i=1,...,m} h_{W_i(g)}(u),
\]

where $h_{S(g)}$ is convex on $\mathbb{R}^6$.
$q$ is invariant under fixed translation of the grasp center and contact positions.
Furthermore, let $\mathbb{B}(r) = \lbrace x \in \mathbb{R}^3 : ||x|| \leq r \rbrace$.
Then $q$ is Lipschitz continuous on grasps with $m$ contact points lying in the set $X = \lbrace (c_1, \dots, c_m,n_1, \dots,n_m,z) : (c_i-z) \in \mathbb{B}(r), n_i \in \mathbb{S}^2 \rbrace$ with a Lipschitz constant given by $L= (1+\mu)(1+r)$ and where we use distance measure 

\[
  d(g,g') = \sum_i ||(c_i-z)-(c_i'-z')|| + \sum_i ||n_i - n_i'||.
\]

We hence have 

\[
|q(g) - q(g')| \leq Ld(g,g'),\  \forall g,g' \in X.
\]

Since $Q(g) = \max(0,q(g))$, $Q$ is also Lipschitz continuous with the same constant $L$ on $X$. 
\end{theorem}

Setting $l_{i,a,b} = h_{W_i(g)}(a,b)$, we have for $||(a,b)|| \leq 1$ that is $|l_{i,a,b}(g) - l_{i,a,b}(g')|$ is bounded by $|\langle n_i,a+b\times (c_i-z)\rangle - \langle n_i',a+b\times(c_i'-z')\rangle|+\mu|||n_i \times (a+b \times (c_i -z))|| - ||n_i \times (a + b \times (c_i' - z')) |||$, using Theorem \ref{lemma35}.
By using the following facts $||a|| \leq 1$, $||b|| \leq 1$, $||v \times w || \leq ||w||||v||$, $|\langle v,w\rangle | \leq ||v||||w||$, we obtain:  

\begin{align*}
|l_{i,a,b}(g) - l_{i,a,b}(g')| &\leq ||n_i - n_i'||(1+||c_i - z||) \\
					&+ ||(c_i - z)-(c_i'-z')|| \\
					&+ \tau(||n_i - n_i'||(1+||c_i - z||)\\
					&+||(c_i - z)-(c_i'-z')||)
\end{align*}

\subsection{Our Extension}
We now define the concept of a probabilistic bound by functions $\sigma_{n_i}(\zeta)$ and $\sigma_{r_i}(\zeta)$, given a probability $\zeta$ they denote the maximum change in grasp parameter from the mean in the distribution for that probability.
Formally, this is defined as:

\[
  \sigma_{n_i}(\zeta)
  =
  \min_{\mathbb{R}_+}d
\]

\[
  s.t. \ \ \left[ \int_{\mathbb{B}(\bar{n}, d)} p(n_i = k) dk \right] \geq \zeta
\]

and 


\[
  \sigma_{r_i}(\zeta)
  =
  \min_{\mathbb{R}_+} d
\]

\[
s.t. \ \ \left[ \int_{\mathbb{B}(\bar{r}, d)} p(r_i = k) dk \right]  \geq \zeta
\]

$\sigma_{n_i}(\zeta)$ references the distribution defined by $p(n_i = k)$.
$r_i = c_i - \bar{z}$ defines a moment arm, where $\bar{z}$ is the expected center of mass, then $\sigma_{r_i}(\zeta)$ references the distribution $p(c_i = k)$.
We now write the above bound as follows: 

 \begin{align*}
|l_{i,a,b}(g) - l_{i,a,b}(g')| &\leq ||\sigma_{n_i}(\zeta)||(1+||\bar{r}_i||) + ||\sigma_{r_i}(\zeta)|| \\
                                      &+ \tau(||\sigma_{n_i}(\zeta)||(1+||\bar{r}_i||)+||\sigma_{r_i}(\zeta)||)
\end{align*}

For convenience we rewrite the bound on a given set of graps parameters as : 

 \begin{align*}
b_i(\zeta) &= ||\sigma_{n_i}(\zeta)||(1+||\bar{r}_i||) + ||\sigma_{r_i}(\zeta)|| \\
                                      &+ \tau(||\sigma_{n_i}(\zeta)||(1+||\bar{r}_i||)+||\sigma_{r_i}(\zeta)||)
\end{align*}

To provide an upper bound overall contact parameters we introduce the following: 

\begin{align}
b(\zeta)  = \max_{i} b_i(\zeta)
\end{align}

To prove if  this bound is preserved on

 \begin{align}
 q(g) = \min_{(a,b) \in \mathbb{R}^6, ||u||=1} \max_{i=1,..,m} l_{i,a,b}(g)
 \end{align}
 
we turn to the general case.
$\lambda(x) = \mbox{inf}_{\alpha \in A} f_\alpha(x)$ and $\lambda(x) = \mbox{sup}_{\alpha \in A} f_\alpha(x)$ are bounded with $b(\zeta)$ if $f_\alpha(x)$ for all $\alpha$ is bounded with $b(\zeta)$ and $\lambda(x)$ is bounded.
Since our bound is invariant to the variables $(a,b)$ and we take the maximum set of parameters $i$.
We can ensure that is true.


\bibliographystyle{ieeetr}
\bibliography{references}

\end{document}
